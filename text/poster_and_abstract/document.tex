% !TeX spellcheck = en_US
\documentclass[11pt,a4paper]{article}


\usepackage[margin=1in]{geometry}
\usepackage{amsmath,amssymb,amsthm}
\usepackage{graphics}
\usepackage{hyperref}
\usepackage{physics}
\usepackage{cite}
\usepackage{todonotes}
\usepackage{comment}
\usepackage{authblk}
\usepackage{optidef}
\usepackage{graphicx}
\usepackage{subcaption}
\usepackage{booktabs}
\captionsetup{subrefformat=parens}

\title{Fixed interval scheduling problem with no idle time with an application to music arrangement problem}
\author[1,2]{Ludmila Botelho}
\author[1]{\"Ozlem Salehi}
\affil[1]{Institute of Theoretical and Applied Informatics, Polish Academy of Sciences, Bałtycka~5, 44-100 Gliwice, Poland}
\affil[2]{Joint Doctoral School, Silesian University of Technology, Akademicka 2a, 44-100 Gliwice, Poland}
\date{}

\begin{document}
	\maketitle
	\begin{abstract}
		With the emergence of quantum computers, a new field of algorithmic for optimization tasks is flourishing. An alternative model of computation is adiabatic quantum computing (AQC), and a heuristic algorithm known as quantum annealing running in the framework of AQC is a promising method for solving optimization problems. With this in mind, we proposed a multiobjective optimization for a fixed interval scheduling problem with application for music reduction.
	\end{abstract}
\end{document}