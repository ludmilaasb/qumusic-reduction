\documentclass[aps,reprint]{revtex4-2}

\usepackage[english]{babel}
\usepackage[utf8x]{inputenc}
\usepackage[a4paper]{geometry}
\usepackage{amsmath}
\usepackage{amssymb}
\usepackage{amssymb}
\usepackage{microtype}
\usepackage{hyperref}
\usepackage{physics}
\usepackage{lipsum}
%%%%%%%%%%%%%%%%%%% TEMPLATE DEFNITION: DO NOT MODIFY! %%%%%%%%%%%%%%%%%%%%%%%%%%

\onecolumngrid % Single column.

\makeatletter
\newcommand{\printentry}{%
    \@author@finish%
    \title@column %
    \titleblock@produce %
    \suppressfloats[t] %
    \let\@AAC@list\@empty %
    \let\@AFF@list\@empty %
    \let\@AFG@list\@empty %
    \let\@address\@empty %
    \titlepage@sw %
    {\vfil\clearpage}%
}
\makeatother

\newcommand{\printheader}{%
    \title{\normalsize Submission to QCMC 2022 \\ %
            International Conference on Quantum Communication,\\ 
            Measurement and Computing 2022}
    \printentry
}

%%%%%%%%%%%%%%%%%%%%%%%%%%%%%%%%%%%%%%%%%%%%%%%%%%%%%%%%%%%%%%%%%%%%%%%%%%%%%%%%%

\begin{document}

\printheader

%%%%%%%%%%%%%%%%%%%%%%%% TO BE FILLED IN BY AUTHORS %%%%%%%%%%%%%%%%%%%%%%%%%%%%%


\title{Fixed interval scheduling problem with no idle time with an application to music arrangement problem} % Please indicate the title with \title
\author{Ludmila Botelho$^{1,2}$, \"Ozlem Salehi$^{1}$} % Please indicate the authors with \author
\affiliation{$^1$Institute of Theoretical and Applied Informatics, Polish Academy of Sciences, Bałtycka~5, 44-100 Gliwice, Poland\\
$^2$Joint Doctoral School, Silesian University of Technology, Akademicka 2a, 44-100 Gliwice, Poland
} % Use \affiliation command to specify the affiliations.

\printentry
\title{\small Abstract}
\begin{abstract}
  % Please include the abstract here.
  

With the emergence of quantum computers, the class of combinatorial optimization problems is one of the potential targets to tackle. In the context of current quantum devices subjected to the Noisy Intermediate-Scale Quantum (NISQ) paradigm, one of the promising algorithms is Quantum Annealing (QA) which runs in the framework of Adiabatic Quantum Computing. QA is used for solving optimization tasks expressed in terms of an Ising Hamiltonian designed in such a way that its ground state encapsulates the optimal solution to the considered problem. Obtaining the Ising Hamiltonian is not always a trivial task and it is often more convenient to express problems in the form of  Quadratic Unconstrained Binary Optimization (QUBO), which can then be expressed in the form of an Ising Hamiltonian.

Among the optimization problems, the class of scheduling problems has significance both for the areas of computer science and operations research. Operational Fixed Interval Scheduling Problem (OFISP)  is characterized as the problem of assigning a number of jobs, each with a fixed starting and end time and a weight, to a number of machines with the restrictions that each machine can handle a single job at a time and preemption is not allowed. The objective is to find an assignment of jobs to the machines with maximal total weight. Now, we introduce another constraint known as no idle-time in the literature so that while the total weight is maximized, the duration that the machines are idle is minimized simultaneously and we obtain a multiobjective problem. We call this version of the problem as Operational Fixed Interval Scheduling Problem with No-Idle Time (OFISP$_\text{n-i}$).   

One possible application to the OFISP$_\text{n-i}$ is the Music Reduction Problem, a process that arranges music for a target instrument by reducing original music preserving the characteristics of the original piece as much as possible. The music is separated in different tracks for each instrument and tracks are segmented in phrases, which are in in the OFISP scenario equivalent to jobs. The phrases contains different priorities measure by their information entropy. The goal is to selected the phrases in such way that it preservers the similarity with the original piece but minimizing the silence between phrases. The number of new tracks for the reduced composition are the equivalent to machines performing the jobs. The problem is expressed using a QUBO formulation by introducing binary variables to represent which job is selected for a time point and the objective is to maximize the total weight of the selected jobs. Additional terms are introduced to make sure that the number of selected jobs at a time is equal to the number of machines and that the idle-time is minimized.


In this paper, we present a QUBO formulation for the OFISP$_\text{n-i}$ in general and provide experimental results for the case of Music Reduction. The problem can be expressed using a QUBO formulation mapping the jobs to the binary variables, maximizing their weights. To keep the no-idle condition additional constraints are introduced. The experimental results are obtained using the commercially available D-Wave Solvers.
\end{abstract}
\printentry

\end{document}
