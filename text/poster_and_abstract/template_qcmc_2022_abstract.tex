\documentclass[aps,reprint]{revtex4-2}

\usepackage[english]{babel}
\usepackage[utf8x]{inputenc}
\usepackage[a4paper]{geometry}
\usepackage{amsmath}
\usepackage{amssymb}
\usepackage{amssymb}
\usepackage{microtype}
\usepackage{hyperref}
\usepackage{physics}

%%%%%%%%%%%%%%%%%%% TEMPLATE DEFNITION: DO NOT MODIFY! %%%%%%%%%%%%%%%%%%%%%%%%%%

\onecolumngrid % Single column.

\makeatletter
\newcommand{\printentry}{%
    \@author@finish%
    \title@column %
    \titleblock@produce %
    \suppressfloats[t] %
    \let\@AAC@list\@empty %
    \let\@AFF@list\@empty %
    \let\@AFG@list\@empty %
    \let\@address\@empty %
    \titlepage@sw %
    {\vfil\clearpage}%
}
\makeatother

\newcommand{\printheader}{%
    \title{\normalsize Submission to QCMC 2022 \\ %
            International Conference on Quantum Communication,\\ 
            Measurement and Computing 2022}
    \printentry
}

%%%%%%%%%%%%%%%%%%%%%%%%%%%%%%%%%%%%%%%%%%%%%%%%%%%%%%%%%%%%%%%%%%%%%%%%%%%%%%%%%

\begin{document}

\printheader

%%%%%%%%%%%%%%%%%%%%%%%% TO BE FILLED IN BY AUTHORS %%%%%%%%%%%%%%%%%%%%%%%%%%%%%


\title{Fixed interval scheduling problem with no idle time with an application to music arrangement problem} % Please indicate the title with \title
\author{Ludmila Botelho$^{1,2}$, \"Ozlem Salehi$^{1}$} % Please indicate the authors with \author
\affiliation{$^1$Institute of Theoretical and Applied Informatics, Polish Academy of Sciences, Bałtycka~5, 44-100 Gliwice, Poland\\
$^2$Joint Doctoral School, Silesian University of Technology, Akademicka 2a, 44-100 Gliwice, Poland
} % Use \affiliation command to specify the affiliations.

\printentry
\title{\small Abstract}
\begin{abstract}
  % Please include the abstract here.
  
With the emergence of quantum computers, a new field of algorithmic for optimization tasks is flourishing. An alternative model of computation is adiabatic quantum computing (AQC), and a heuristic algorithm known as Quantum Annealing (QA) running in the framework of AQC is a promising method for solving optimization problems. 

QA is used for solving optimization tasks expressed in term of a Ising Hamiltonian designed in such a way that its ground state encapsulates the optimal solution to the considered problem. Quadratic Unconstrained Binary Optmization (QUBO),

We demonstrate the use of QA to solve ???? and we evaluate its efficiency and accuracy for this problem
from empirical results. Our approach uses the commercial quantum annealer available from D-Wave Systems to
implement several hard constraints. The D-Wave 2000Q is a commercially available quantum annealing device... By reducing ???? to QUBO form and then embedding this problem into the D-Wave processor, we use QA to reduce a complex music sheet to simpler version, a problem known as Music Arrange Reduction. 
\end{abstract}
\printentry

\end{document}
