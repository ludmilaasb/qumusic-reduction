\documentclass[aps,reprint]{revtex4-2}

\usepackage[english]{babel}
\usepackage[utf8x]{inputenc}
\usepackage[a4paper]{geometry}
\usepackage{amsmath}
\usepackage{amssymb}
\usepackage{amssymb}
\usepackage{microtype}
\usepackage{hyperref}
\usepackage{physics}
\usepackage{lipsum}
%%%%%%%%%%%%%%%%%%% TEMPLATE DEFNITION: DO NOT MODIFY! %%%%%%%%%%%%%%%%%%%%%%%%%%

\onecolumngrid % Single column.

\makeatletter
\newcommand{\printentry}{%
    \@author@finish%
    \title@column %
    \titleblock@produce %
    \suppressfloats[t] %
    \let\@AAC@list\@empty %
    \let\@AFF@list\@empty %
    \let\@AFG@list\@empty %
    \let\@address\@empty %
    \titlepage@sw %
    {\vfil\clearpage}%
}
\makeatother

\newcommand{\printheader}{%
    \title{\normalsize Submission to QCMC 2022 \\ %
            International Conference on Quantum Communication,\\ 
            Measurement and Computing 2022}
    \printentry
}

%%%%%%%%%%%%%%%%%%%%%%%%%%%%%%%%%%%%%%%%%%%%%%%%%%%%%%%%%%%%%%%%%%%%%%%%%%%%%%%%%

\begin{document}

\printheader

%%%%%%%%%%%%%%%%%%%%%%%% TO BE FILLED IN BY AUTHORS %%%%%%%%%%%%%%%%%%%%%%%%%%%%%


\title{Fixed interval scheduling problem with no idle time with an application to music arrangement problem} % Please indicate the title with \title
\author{Ludmila Botelho$^{1,2}$, \"Ozlem Salehi$^{1}$} % Please indicate the authors with \author
\affiliation{$^1$Institute of Theoretical and Applied Informatics, Polish Academy of Sciences, Bałtycka~5, 44-100 Gliwice, Poland\\
$^2$Joint Doctoral School, Silesian University of Technology, Akademicka 2a, 44-100 Gliwice, Poland
} % Use \affiliation command to specify the affiliations.

\printentry
\title{\small Abstract}
\begin{abstract}
  % Please include the abstract here.
  
With the emergence of quantum computers, a new field of algorithmic for optimization tasks is flourishing. In the context of current devices, subjected to the Noisy Intermediate-Scale Quantum (NISQ) paradigm, one of the promising models for quantum computation is the Adiabatic Quantum Computing (AQC), and a heuristic algorithm known as Quantum Annealing (QA) running in the framework of AQC. QA is used for solving optimization tasks expressed in term of a Ising Hamiltonian designed in such a way that its ground state encapsulates the optimal solution to the considered problem. It is often more convenient to express problems in the form of Quadratic Unconstrained Binary Optimization (QUBO), which can then be expressed in the form an Ising Hamiltonian.

Among the optimization problems, the class of job scheduling has both significance for computational science and operations research. We investigate the Operational Fixed Interval Scheduling Problem (OFISP) which is characterized as the problem of assigning a number of jobs, each with a fixed starting and end time and a weight, to a list of machines with the restrictions that each machine can handle a single job at a time and preemption is not allowed. The objective is to find an assignment of jobs to machines with maximal total weight. The problem has been studied by many researchers can be solved in polynomial time. Now we introduce another constraint known as the idle-time in the literature, so that while maximizing the total weight, we also try to minimize the duration that the machines are idle at the same time, obtaining an multiobjective problem. We call this version of the problem as Operational Fixed Interval Scheduling Problem with No-Idle Time (OFISP$_\text{n/i}$).   

One possible application to the OFISP$_\text{n/i}$ is the Music Reduction Problem, a process that arranges music for a target instrument by reducing original music preserving the characteristics of the original piece as much as possible. The music is separated in different tracks for each instrument and tracks are segmented in phrases, which are in in the OFISP scenario equivalent to jobs. The phrases contains different priorities measure by their information entropy. The goal is to selected the phrases in such way that it preservers the similarity with the original piece but minimizing the silence between phrases. The number of new tracks for the reduced composition are the equivalent to machines performing the jobs. The problem is expressed using a QUBO formulation by introducing binary variables to represent which job is selected for a time point and the objective is to maximize the total weight of the selected jobs. Additional terms are introduced to make sure that the number of selected jobs at a time is equal to the number of machines and that the idle-time is minimized.

In this paper, we present QUBO formulation for the OFISP problem in general and provide experimental results for the case of Music Reduction. Our approach uses the commercial quantum annealer available from D-Wave Systems to implement several hard constraints.
\end{abstract}
\printentry

\end{document}
