% !TeX spellcheck = en_US
\documentclass[11pt,a4paper]{article}


\usepackage[margin=1in]{geometry}
\usepackage{amsmath,amssymb,amsthm}
\usepackage{graphics}
\usepackage{hyperref}
\usepackage{physics}
\usepackage{cite}
\usepackage{todonotes}
\usepackage{comment}
\usepackage{authblk}
\usepackage{optidef}
\usepackage{graphicx}
\usepackage{subcaption}
\usepackage{booktabs}
\captionsetup{subrefformat=parens}

\title{}
\author[1]{Ludmila Botelho}
\author[1]{\"Ozlem Salehi}
\affil[1]{Institute of Theoretical and Applied Informatics, Polish Academy of Sciences, Bałtycka~5, 44-100 Gliwice, Poland}
\date{}


\begin{document}
	\maketitle
	\begin{abstract}

\end{abstract}


\section{Formulations}

Suppose that there are $ N $ tracks in the song and the goal is to reduce the number of tracks to $ M $.
\subsection{First approach}

This is the approach used to model fixed job scheduling \cite{arkin1987scheduling, barcia2005k}. 

Let $L $ be the number of phrases. Let $ p_i $ be the binary variables such that $ p_i=1 $ if phrase $i$ is selected and 0 otherwise for $ i=1, \dots ,L $. Let us denote the set of phrases $ p_1,p_2,\dots,p_L $ by $ P $. Let $ G=(P,E) $ be the graph with vertices $ P $ such that edge $(i,j) \in E$ if phrases $ i $ and $ j $ are compatible i.e. they do not intersect.

Let $ x_{ij} $ be the assignment variable such that $ x_{ij} = 1 $ if phrase $i$ is assigned to track $j$ for $ i = 1, \dots ,L $ and $ j=1, \dots ,M  $. $ e(p_i) $ denotes the entropy for phrase $ p_i $. The goal is to maximize 
\begin{equation}
\sum_{i=1}^L e(p_i)p_i	
 \end{equation}
 such that
 \begin{align}
 &\sum_{j=1}^M x_{ij} \leq 1 \mbox{ for }i=1\dots L \\
 &\sum_{i \in I} x_{ij} \leq 1 \mbox{ for } I \subseteq P
 \end{align}
 where $ I $ in any maximal independent set of $ G $.
\end{document}

The first constraint ensures that each phrase is assigned to at most one track. The second constraints ensures that incompatible phrases are not selected. In our case, $ G $ is an interval graph and the problem is solvable in time $ O(L^2 \log L) $ by a classical algorithm. The number of variables required by this approach is $ O(LM) $. Note that $ L $ depends on the number of tracks and $ L = O(N) $. 

The drawback of this formulation is that it does not take into consideration if after the reduction some tracks are empty in a given time period. It can even be the case that all tracks are empty at a time point. In terms of jobs and machines, this corresponds to machine being idle for some time. 

